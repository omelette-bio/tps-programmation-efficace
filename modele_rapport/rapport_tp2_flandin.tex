\documentclass{rapport}
\usepackage[utf8]{inputenc}

\usepackage{pifont} % Pour les symboles appelés par la macro \ding
\usepackage{url} % Comme son nom l'indique, pour les url...

\usetikzlibrary{positioning} % Bibliothèque tikz pour positionner des nœuds relativement à d'autres

\usepackage[colorlinks, citecolor=red!60!green, linkcolor=blue!60!green, urlcolor=magenta]{hyperref} % Pour que les liens soient cliquables. Les options permettent de mettre les liens en couleur.

\usepackage{algorithm}
\usepackage{algo}
\usepackage{colorationSyntaxique}
\usepackage{listings}
\usepackage{xcolor}



% Pour un rapport en français 
\usepackage[francais]{babel} % Commenter pour un rapport en anglais
\renewcommand\bibsection{\section*{Bibliographie}} % Commenter pour un rapport en anglais

% \englishTitlePage % Décommenter pour une page de titre en anglais


\pagestyle{fancy}
\renewcommand{\sectionmark}[1]{\markboth{\thesection.\ #1}{}}
\fancyfoot{}

\fancyhead[LE]{\textsl{\leftmark}}
\fancyhead[RE, LO]{\textbf{\thepage}}
\fancyhead[RO]{\textsl{\rightmark}}

\def\Latex{\LaTeX\xspace}
\def\etc{\textit{etc.}\xspace}

\lstset{
    %language=C,                          % Set the programming language
    basicstyle=\ttfamily\small,          % Font style and size for the code
    keywordstyle=\color{blue}\bfseries,  % Keywords in bold blue
    %identifierstyle=\color{green!50!black},       % Variables and identifiers in black
    commentstyle=\color{gray}\itshape,   % Comments in italic gray
    stringstyle=\color{red},             % Strings in red
    numberstyle=\tiny\color{gray},       % Line numbers in small gray text
    directivestyle=\color{purple},       % Preprocessor directives in purple
    emphstyle=\color{teal},              % Emphasized words in teal
    numbers=left,                        % Line numbers on the left
    numbersep=5pt,                       % Space between code and line numbers
    %frame=single,                        % Frame around the code
    rulecolor=\color{black},             % Frame color
    breaklines=true,                     % Enable line breaking
    %backgroundcolor=\color{gray!10},     % Light gray background
    showstringspaces=false,              % Hide spaces in strings
    morekeywords={uint8_t, uint16_t},    % Add custom keywords
    literate={->}{{$\to$}}2              % Replace '->' with a right arrow
}



\title{Transformations de boucles}
\author{Francois Flandin}
\supervisor{Pr Sid Touati}
\date{Premier semestre de l'annee 2024-2025}

% \universityname{Université Côte d'Azur} % Nom de l'université.
\type{TP} % Type de document
% \formation{Master Informatique} % Nom de la formation

% Retrouver les autres options possibles dans le document rapport.pdf

\begin{document}

  \maketitle

  \clearpage
  \tableofcontents

  \clearpage
  \section{Deroulage de boucle}
  \subsection{Code du deroulage}
  \begin{lstlisting}[language=C]
    for (k=0; k< P; k+=8)
    {
      C[i][j] = C[i][j] + A[i][k] * B[k][j];
      C[i][j] = C[i][j] + A[i][k+1] * B[k+1][j];
      C[i][j] = C[i][j] + A[i][k+2] * B[k+2][j];
      C[i][j] = C[i][j] + A[i][k+3] * B[k+3][j];
      C[i][j] = C[i][j] + A[i][k+4] * B[k+4][j];
      C[i][j] = C[i][j] + A[i][k+5] * B[k+5][j];
      C[i][j] = C[i][j] + A[i][k+6] * B[k+6][j];
      C[i][j] = C[i][j] + A[i][k+7] * B[k+7][j];
    }
  \end{lstlisting}

  \subsection{Gain de performances apres deroulage source}
  \subsubsection*{Temps d'execution du programme initial}
  \begin{lstlisting}[language=sh]
    4.40s user      0.03s system      4.441 total
    4.68s user      0.02s system      4.720 total
    4.67s user      0.03s system      4.711 total
  \end{lstlisting}
  Moyenne des temps d'execution : 4.624s.

  \newline\newline
  \subsubsection*{Temps d'execution apres deroulage source}
  \begin{lstlisting}[language=sh]
    4.22s user      0.04s system      4.267 total
    4.37s user      0.03s system      4.411 total
    4.25s user      0.04s system      4.302 total
  \end{lstlisting}
  Moyenne des temps d'execution : 4.326s.
  \newline

  On constate donc un gain de performances de 0.298s avec le deroulage.

  \subsection{Essai avec -funroll-loops}
  \subsubsection*{Resultats}
  \begin{lstlisting}[language=sh]
    4.38s user      0.03s system      4.423 total
    4.41s user      0.02s system      4.446 total
    4.27s user      0.02s system      4.303 total
  \end{lstlisting}
  La moyenne est de 4.390s, il n'y a pas de reel gain de temps par rapport au deroulage source.

  \section{Fusion des boucles}
  \subsection{Code du deroulage de la boucle i}
  \begin{lstlisting}[language=C]
    for (i=0; i< N; i+=3)
    {
      for (j=0; j< M; j++)
        for (k=0; k<P; k++)
          C[i][j] = C[i][j] + A[i][k] * B[k][j];
      
      if (i+1 < N){
        for (j=0; j< M; j++)
          for (k=0; k<P; k++)
            C[i+1][j] = C[i+1][j] + A[i+1][k] * B[k][j];
      }
      if (i+2 < N) {
        for (j=0; j< M; j++)
          for (k=0; k<P; k++)
            C[i+2][j] = C[i+2][j] + A[i+2][k] * B[k][j];
      }
    }
  \end{lstlisting}
  \subsection{Fusion des boucles exterieures}
  On peut donc fusionner ces trois boucles, ce qui donne le resultat suivant:
  \begin{lstlisting}[language=C]
    for (i = 0; i < N; i += 3) {
      for (j = 0; j < M; j++) {
        for (k = 0; k < P; k++)
          C[i][j] = C[i][j] + A[i][k] * B[k][j];
        if (i + 1 < N)
          for (k = 0; k < P; k++)
            C[i+1][j] = C[i+1][j] + A[i+1][k] * B[k][j];
        if (i + 2 < N)
          for (k = 0; k < P; k++)
            C[i+2][j] = C[i+2][j] + A[i+2][k] * B[k][j];
      }
    }
  \end{lstlisting}



 % \begin{lstlisting}\end{lstlisting}


  \end{document}
