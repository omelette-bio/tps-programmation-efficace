\documentclass{rapport}
\usepackage[utf8]{inputenc}

\usepackage{pifont} % Pour les symboles appelés par la macro \ding
\usepackage{url} % Comme son nom l'indique, pour les url...

\usetikzlibrary{positioning} % Bibliothèque tikz pour positionner des nœuds relativement à d'autres

\usepackage[colorlinks, citecolor=red!60!green, linkcolor=blue!60!green, urlcolor=magenta]{hyperref} % Pour que les liens soient cliquables. Les options permettent de mettre les liens en couleur.

\usepackage{algorithm}
\usepackage{algo}
\usepackage{colorationSyntaxique}
\usepackage{listings}


% Pour un rapport en français 
\usepackage[francais]{babel} % Commenter pour un rapport en anglais
\renewcommand\bibsection{\section*{Bibliographie}} % Commenter pour un rapport en anglais

% \englishTitlePage % Décommenter pour une page de titre en anglais


\pagestyle{fancy}
\renewcommand{\sectionmark}[1]{\markboth{\thesection.\ #1}{}}
\fancyfoot{}

\fancyhead[LE]{\textsl{\leftmark}}
\fancyhead[RE, LO]{\textbf{\thepage}}
\fancyhead[RO]{\textsl{\rightmark}}

\def\Latex{\LaTeX\xspace}
\def\etc{\textit{etc.}\xspace}

\lstset{                  % Specify language
    basicstyle=\ttfamily\small,     % Code font and size
    keywordstyle=\color{blue},      % Color for keywords
    commentstyle=\color{gray},      % Color for comments
    stringstyle=\color{red},        % Color for strings
    numbers=left,                   % Add line numbers
    numberstyle=\tiny\color{gray},  % Style for line numbers
    % frame=single,                   % Add a border around code
    breaklines=true,                % Line wrapping
    % backgroundcolor=\color{gray!10} % Light gray background
}


\title{Évaluation des performances d’un programme et optimisations de
code par compilation}
\author{Francois Flandin}
\supervisor{Pr Sid Touati}
\date{Premier semestre de l'annee 2024-2025}

% \universityname{Université Côte d'Azur} % Nom de l'université.
\type{TP} % Type de document
% \formation{Master Informatique} % Nom de la formation

% Retrouver les autres options possibles dans le document rapport.pdf

\begin{document}

  \maketitle

  \clearpage
  \tableofcontents

  \clearpage

  \section{Mesures des performances et profilage d’un programme}
  
  \subsection{Temps d'execution des programmes}

  \subsubsection{Multiplication de matrices}
  \textbf{version statique}
  \begin{lstlisting}
    23.05 user      0.02 system      0:23.11 elapsed
    23.33 user      0.02 system      0:23.39 elapsed
    23.45 user      0.01 system      0:23.51 elapsed
    24.09 user      0.02 system      0:24.19 elapsed
  \end{lstlisting}
  On remarque que le temps d'execution reste stable. 
  Presque 99.5\% du temps d'execution sert pour le code utilisateur, tandis que 0.001\% du temps sert pour les appels systeme. 
  Le reste correspond aux "pauses" du processeur.
  \newline
  \newline
  \textbf{version dynamique}
  \begin{lstlisting}
    27.01 user      0.02 system      0:27.16 elapsed
    27.59 user      0.04 system      0:27.81 elapsed
    27.72 user      0.03 system      0:27.89 elapsed
    28.28 user      0.02 system      0:28.46 elapsed
  \end{lstlisting}

  \subsubsection{Multiplication d'un vecteur par un scalaire}
  \textbf{version statique}
  \begin{lstlisting}
    0.45 user       0.10 system      0:00.56 elapsed
    0.43 user       0.11 system      0:00.55 elapsed
    0.41 user       0.10 system      0:00.52 elapsed
    0.43 user       0.08 system      0:00.52 elapsed
  \end{lstlisting}

  \newline
  \newline
  \textbf{version dynamique}
  \begin{lstlisting}
    0.41 user       0.09 system      0:00.51 elapsed
    0.39 user       0.10 system      0:00.49 elapsed
    0.40 user       0.08 system      0:00.49 elapsed
    0.40 user       0.08 system      0:00.49 elapsed
  \end{lstlisting}

  \subsubsection{Addition de deux vecteurs}
  \textbf{version statique}
  \begin{lstlisting}
    0.84 user       0.55 system      0:01.40 elapsed
    0.82 user       0.58 system      0:01.41 elapsed
    0.85 user       0.55 system      0:01.41 elapsed
    0.81 user       0.53 system      0:01.35 elapsed
  \end{lstlisting}
  \newline
  \newline
  \textbf{version dynamique}
  \begin{lstlisting}
    0.91 user       0.58 system      0:01.51 elapsed
    0.89 user       0.54 system      0:01.44 elapsed
    0.86 user       0.55 system      0:01.42 elapsed
    0.86 user       0.53 system      0:01.40 elapsed
  \end{lstlisting}
  
  \subsection{Comment calculer l'IPC, le CPI et le GFLOPS}
  
  \subsubsection{Calculer l'IPC}
  On peut utiliser la commande \textit{perf stat \<nom\_prog\>} qui nous donne ceci avec \textit{mat\_mult\_STATIC}:
  \begin{lstlisting}
    94,108,593,606      cycles:u            #    1.594 GHz                 
    264,116,293,661     instructions:u      #    2.81  insn per cycle
  \end{lstlisting}

  \subsubsection{Calculer le CPI}
  La formule pour calculer le CPI est la suivante :
  \[
    CPI = \frac{nombre\_de\_cycles}{nombre\_d'instructions}
  \]

  Avec la commande precedente, on a le nombre de cycles et le nombre d'instructions de l'execution d'un programme, on a donc 
  \[
    \frac{94,108,593,606}{264,116,293,661} \approx 0.35
  \]
  
  \subsubsection{Calculer le GFLOPS}
  La formule pour calculer le GFLOPS est la suivante:
  \[
    GFLOPS = \frac{Nombre\_de\_calculs\_flottants}{Temps\_d'execution}
  \]
    
  \section{Optimisations de code avec le compilateur gcc}
  \subsection{-fprofile-generate}
  Voici ce qu'en dit le manuel de gcc:
  \begin{lstlisting}
  Enable options usually used for instrumenting application to produce profile useful for later recompilation with profile feedback based optimization. You must use -fprofile-generate both when compiling and when linking your program.
  \end{lstlisting}

  \section{Optimisations de code avec le compilateur d’Intel icc ou icx}
  \subsection{Multiplication de matrices}
  \textbf{version statique}
  \begin{lstlisting}
    13.84 user      0.02 system      0:13.88 elapsed
    13.52 user      0.01 system      0:13.57 elapsed
    15.97 user      0.02 system      0:16.02 elapsed
    15.98 user      0.01 system      0:16.03 elapsed 
  \end{lstlisting}

  \textbf{version dynamique}
  \begin{lstlisting}
    15.95 user      0.02 system      0:16.07 elapsed
    13.64 user      0.02 system      0:13.77 elapsed
    14.68 user      0.03 system      0:14.87 elapsed 
    15.58 user      0.04 system      0:15.80 elapsed
  \end{lstlisting}

  \subsection{Multiplication d'un vecteur par un scalaire}
  \textbf{version statique}
  \begin{lstlisting}
    0.00 user       0.11 system      0:00.12 elapsed 
    0.01 user       0.10 system      0:00.12 elapsed
    0.02 user       0.10 system      0:00.12 elapsed
    0.01 user       0.09 system      0:00.11 elapsed
  \end{lstlisting}


  \textbf{version dynamique}
  \begin{lstlisting}
    0.01 user       0.08 system      0:00.10 elapsed 
    0.02 user       0.08 system      0:00.10 elapsed
    0.01 user       0.09 system      0:00.11 elapsed 
    0.01 user       0.10 system      0:00.11 elapsed
  \end{lstlisting}

  \subsection{Addition de vecteurs}
  \textbf{version statique}
  \begin{lstlisting}
    0.10 user       0.58 system      0:00.69 elapsed
    0.10 user       0.56 system      0:00.68 elapsed
    0.11 user       0.60 system      0:00.72 elapsed
    0.09 user       0.59 system      0:00.70 elapsed
  \end{lstlisting}


  \textbf{version dynamique}
  \begin{lstlisting}
    0.12 user       0.58 system      0:00.71 elapsed
    0.13 user       0.61 system      0:00.74 elapsed
    0.11 user       0.65 system      0:00.77 elapsed
    0.12 user       0.62 system      0:00.75 elapsed
  \end{lstlisting}

  \end{document}
